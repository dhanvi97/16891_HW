\documentclass[11pt]{article}
\usepackage{latexsym}
\usepackage{amsmath}
\usepackage{amssymb}
\usepackage{amsthm}
\usepackage{epsfig}
\usepackage[tight]{subfigure}
\usepackage{caption}
\usepackage{booktabs}
\usepackage{graphicx}

\usepackage{amsmath}

\DeclareMathOperator*{\minimize}{min}
\DeclareMathOperator*{\maximize}{max}

\usepackage{algorithm}
 %on linux you may need to run sudo apt-get install texlive-full to install algorithm.sys
\usepackage{algorithmic}

\usepackage{verbatim}

\newcommand{\handout}[6]{
  \noindent
  \begin{center}
  \framebox{
    \vbox{
      \hbox to 5.78in { {#1} \hfill #2 }
      \vspace{4mm}
      \hbox to 5.78in { {\Large \hfill #5  \hfill} }
      \vspace{1mm}
      \hbox to 5.78in { {\Large \hfill #6  \hfill} }
      \vspace{2mm}
      \hbox to 5.78in { {\em #3 \hfill #4} }
    }
  }
  \end{center}
  \vspace*{4mm}
}

\newcommand{\lecture}[6]{\handout{#1}{#2}{#3}{#4}{#5}{#6}}
\newcommand{\collision}[0]{\mathrm{collision}}
\newcommand{\nocollision}[0]{\overline{\collision}}

\newcommand*{\QED}{\hfill\ensuremath{\square}}

\newtheorem{theorem}{Theorem}
\newtheorem{corollary}[theorem]{Corollary}
\newtheorem{lemma}[theorem]{Lemma}
\newtheorem{observation}[theorem]{Observation}
\newtheorem{proposition}[theorem]{Proposition}
\newtheorem{definition}[theorem]{Definition}
\newtheorem{claim}[theorem]{Claim}
\newtheorem{fact}[theorem]{Fact}
\newtheorem{assumption}[theorem]{Assumption}
\newtheorem{note}[theorem]{Note}

% 1-inch margins, from fullpage.sty by H.Partl, Version 2, Dec. 15, 1988.
\topmargin 0pt
\advance \topmargin by -\headheight
\advance \topmargin by -\headsep
\textheight 8.9in
\oddsidemargin 0pt
\evensidemargin \oddsidemargin
\marginparwidth 0.5in
\textwidth 6.5in

\parindent 0in
\parskip 1.5ex
%\renewcommand{\baselinestretch}{1.25}

\title
{ 
{\small Multi Robot Planning and Coordination (16-891) Spring 2024 - The Robotics Institute} \\
Homework 1 : Multi-Agent Path Finding (MAPF) \\
{\small Instructor: Jiaoyang Li \;\;\; TA: Dhanvi Sreenivasan }\\
}
\author{Your Name Here}

\begin{document}

\maketitle

\textbf{Collaborators: }
%Everyone you collaborated with and a short description of what you discussed
\begin{enumerate}
    \item 
\end{enumerate}

\section{Task 1: Implementing Joint-State A* and Space-Time A*}

\textbf{(1 points) Q - In practice, wheeled robots often follow the turn-and-move paradigm, where, at each time step, they can either wait, rotate $90^o$ clockwise, rotate $90^o$ counter-clockwise, or move forward by one cell in the direction they are pointing. Given a fleet of $N$ such robots, what is the maximum branching factor for the search tree of joint-state A*?}

 A - % Answer here

\section{Task 2: Implementing Prioritized Planning}

\textbf{(2 points) Q - Document the changes you made in your code to handle goal related edge-cases and the failure condition. Also, state and justify the upper bound you used in your code.}

A - % Answer here

\textbf{(2 points) Q - Show that Prioritized Planning is Incomplete and Suboptimal}

A - 
\newpage
\section{Task 3: Conflict-Based Search}

\textbf{(1 points) Q - In the pseudocode for the CBS Algorithm, notice that line \ref{alg:cbs:collPicker} doesn't give you much information about which conflict to pick for resolution.
\begin{itemize}
    \item Does the strategy of picking a conflict affect the solution cost? 
    \item Does the strategy of picking a conflict affect the computation time of the solution? 
    \item List three strategies that one could employ to pick conflicts
\end{itemize}}
A - 

\section{Task 5:  Challenge Your MAPF Solvers}

\textbf{(2 points) Q - Document your observation on their CPU time (seconds) and the sum of costs for each of the solvers on both  the maps. Explain the differences you see between them. }

A -
 

\end{document} % Done!